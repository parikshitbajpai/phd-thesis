\chapter{Conclusion}
\label{conclusion}

	Recent trends in modelling and simulation of various materials have adopted the multiscale approach wherein the information from smaller scales is used to develop and improve the models at larger scale in order to better simulate material behaviour. Nuclear materials, in particular, stand to benefit from these multiscale, multiphysics simulation models and have therefore driven the development of advanced modelling and simulation tools. The Multiphysics Object Oriented Programming Environment \texttt{(MOOSE)} developed by the Idaho national Laboratory enables such high fidelity simulations of nuclear materials and contains dedicated applications to model the behaviour of nuclear materials at various scales. However, it has, until now, lacked an application that can model corrosion at the meso-scale and efforts are now underway to develop a new application called \texttt{Yellowjacket}. An essential requirement of many nuclear material simulations is the ability to reliably predict the material properties as material composition evolves and it requires thermodynamic equilibrium calculations to predict the phase distribution at given temperatures and pressure. Particularly, for the case of corrosion modelling, the chemical potentials of various species that exist in the system acts as the driving force for corrosion. Therefore, there is considerable interest in integrating thermodynamic equilibrium computations in \texttt{MOOSE} and specifically \texttt{Yellowjacket}.
	
	The proposed research is aimed at developing a thermodynamic equilibrium solver to predict the phase assemblage of a multicomponent system for a given composition at isothermal, isobaric conditions. Through advanced algorithm development and efficient implementation of performance enhancing strategies, this research will focus on accelerating the performance of thermodynamic computations which are inherently very complex and can significantly impede the computational performance in coupled multiphysics codes. The need to minimise computation time while ensuring that the highly non-linear and non-convex system satisfies the conditions of thermodynamic equilibrium results in a challenging global optimisation problem. Over the course of this work, the Gibbs energy minimisation approach for computing thermodynamic equilibrium will be coupled with the proposed performance enhancing strategies and global optimisation schemes to efficiently and reliably predict the thermodynamic  equilibrium in large multicomponent systems. 
	
	The research will enable coupling thermodynamic equilibrium calculations within multiphysics codes and enable high fidelity material and process simulations to support the development of advanced nuclear reactors.
	
	
	
	