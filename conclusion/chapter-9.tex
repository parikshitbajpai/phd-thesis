\chapter{Recommendations} \label{chap:future}
	Several improvements can be made to the software to extend the range of applications and to enhance the computational performance. Currently, explicit expressions of Gibbs energy and chemical potential are required for each model. Implementing the pre-derived expressions allows reducing the computational expenses but also has a number of limitations. First, not-only is significant effort is required to implement new models but doing so can inevitably introduce typographical errors creating bugs which might be tedious to debug. Second, every time a new derivative is required, such as with respect to temperature to calculate heat capacity, the user or developer must invest significant time. With automatic differentiation methods maturing, their impact on performance has significantly reduced and their implementation is not as big an impediment as it was until a few years ago. Using MetaPhysicL \cite{Lindsay:2021aa}, these disadvantages of hard-coded expressions can be overcome and by strategically invoking this capability, negligible performance impact can be achieved. Another recommendation is efforts to reduce the computational cost of equilibrium calculations. There is a foreseeable advantage in improving the initialisation strategy and one such effort has been partly tried in this work. This idea is based on caching the previous calculations and using the nearest previous calculation to initialise a new calculation. Each calculation can be indexed as nodes of a $k$-dimensional tree ($k$-d tree) and the strategy has already been implemented in MOOSE albeit not tested for {\GEM} despite the data-structures being designed for such application. Though one must consider the impact on memory requirements and search time, this strategy is promising and should be tested and fully implemented in \GEM. There's also a potential on further improving global optimisation algorithms. The convergence of sBB may be significantly improved by implementing heuristics and improved bound tightening strategies specific to the nature of the problem explored here. Similarly, further optimising the hyper-parameters used in APSO may allow improvements to both its performance and reliability helping improve the performance of {\GEM} as well.
	
	In terms of real applications, the code may benefit with the addition of several other models which are often used in modelling nuclear materials. The magnetic contribution to Gibbs energies was neglected in this work. Also, less-frequently used models such as the sub-ionic liquid model (SUBI) were not implemented. Implementing these models will allow much broader applications of code. In the same light, implementation of parsing capability for ThermoCalc format (*.tdb) data files might be of interest since many databases, like TAF-ID, are often only available in this format and the tools for converting one format to another are still nascent and bug-riddled. Despite the best attempts at reducing the computational cost of equilibrium calculations, they are inherently expensive and can often impede and even prohibit any real-world simulation. Several strategies can be considered with the goal of accelerating equilibrium calculations in multiphysics simulations. The most promising and simplest to implement would be using the previously mentioned $k$-d tree to interpolate the results in regions with high confidence without running a full equilibrium calculations. As an example, if a new composition lies in a solution phase region, its properties can be estimated reasonably well with lever rule, saving the computational cost. In doing so, one may even look at more advanced strategies such as use of surrogate models.
 
    Though, the code developed as part of this work is a significant contribution as it is, the potential of growth is significant and some of the most promising applications might only be explored in future. Implementing some of the recommendation will vastly improve the performance and capability of the equilibrium code and allow it to achieve a lot more than it already can.