\chapter{Statement of Work} \label{chap:overview}
	
\section{Problem Statement}
	Facilitating the discovery and design of nuclear materials for innovative nuclear materials requires multiscale, multiphysics models and simulations and equilibrium thermodynamics forms an integral part of such simulations by calculating phase stability and other thermodynamic quantities required for many physical phenomena. The main motivation behind this research is to create native computational thermodynamics capability for the multiphysics framework MOOSE. The computational thermodynamics solver allows replacing empirical models of material properties with calculations rooted in laws of thermodynamics and allows one to capture the effect of system evolution on material properties and, by incorporating equilibrium thermodynamics in multiphysics simulations, an unprecedented level of sophistication can be achieved.

	Achieving this objective, however, comes with several challenges. Equilibrium thermodynamic calculations are intrinsically very expensive and adding such calculations to any multiphysics simulation can massively increase the amount of time and resources required to solve the multiphysics problem. Not only is the cost of equilibrium calculations among the highest in the different physics involved in nuclear material simulations but it also scales very rapidly with the size of the system. In fact, the computational cost scales cubically $\left(\mathit{O}(C^3)\right)$ as the number of components $C$ in the system is increased and has the potential of making large simulations practically infeasible. This means that any thermodynamics code included in multiphysics simulations must be carefully designed to reduce the overheads and minimise the computational cost when possible. However, in doing so the reliability of such calculations can not be comprised since this can have obvious safety implications. Another challenge that is encountered in computational thermodynamics is the dependence on thermodynamic databases wherein a poorly performed thermodynamic treatment can lead to spurious results that can be far from reality. While this is a challenge that cannot be addressed within a computational thermodynamics solver, having the capability of incorporating data from multiple sources and building flexibility in the code to adapt to changes in thermodynamic modelling can help mitigate some of these problems. Lastly, integration into multiphysics simulation is itself challenging because different simulations can have different requirements and the interface between thermodynamics and other physical models must be accommodating to the varying requirements. As an example, the thermodynamic quantities needed in a phase field simulation is often significantly different from those in source term analyses and the same interface to equilibrium calculation should be pluggable into both types of simulations without the need of significant redevelopment.
	
	While several codes with the capabilities developed as part of {\GEM} already exist, the rationale behind developing a new equilibrium thermodynamics code lies in the numerous limitations of these pre-existing tools. As all but a handful of these codes are standalone codes aimed at single calculations, phase diagram generation and database development,  the computational time is not of high importance and there has been an insignificant interest in improving their computational performance. Moreover, in order to maintain numerical stability, most of the current codes restrict the number of system components but the intended applications mean that systems with a very large number of components must also be handled. Among the codes that do exist with the capability of handling large systems and are aimed at multiphysics simulations, there are still several drawbacks. Thermochimica, for example, is well suited for incorporating in multiphysics codes but it doesn't meet the MOOSE software quality assurance (SQA) requirements and often the interface must be redeveloped based on specific applications. The procedural approach adopted in the development also makes reusing the existing code to adapt to new models harder and adding new capabilities can require significant developmental effort. A couple of other examples of codes falling into this category are OpenCalphad and PyCalphad and have similar considerations as Thermochimica.  

\section{Objectives}
	The overarching objective of this work is to incorporate native computational thermodynamics capabilities into the multiphysics framework MOOSE through the development of a new state-of-the-art Gibbs energy minimiser called \GEM. The software consolidates and integrates years of advancements and research in computational thermodynamics and leverages state-of-the-art nonlinear equation solvers and optimisation algorithms to enable real-time thermodynamic property calculation in multiphysics simulations. 
	
	Though the most substantial contribution of this work is the creation of the software from grounds up, there are additional contributions made to the field. First, while several algorithms from the literature are used, their implementation has a significant impact on the computational performance and the implementation had to be cognisant of this fact and the implementations in the literature were modified to reflect the evolution in computing architectures. Seconds, despite years of research into model and database development, it was realised that knowledge gaps exist in the understanding of many models, such as the modified quasichemical model (MQM) and this worked helped in improving the understanding of such models. Third, there is often a lack of understanding about the algorithms, their capabilities and limitations, and the interpretation of the results from equilibrium thermodynamics softwares and, through this work, an effort was made to clarify some these choices. Lastly, despite year of research into global optimisation algorithms, a majority of current equilibrium codes use a rather simplistic sampling approach that often leads to failure in converge or inaccurate results. This work uses an experimental approach to quantitatively compare some promising global optimisation approaches in terms of their applicability to the phase equilibria problem. 
	
	Thus, while the creation of {\GEM} in itself counts as an advancement in the field of computational thermodynamics, several contributions extending beyond the software tools have also been made. The major contributions of the work can be summarised as follows:
	\begin{enumerate}
		\item Development of a new MOOSE-based Gibbs energy minimiser written in C++ with support for several thermodynamic models used in nuclear material modelling. Since the code is currently export controlled, it has not been included here but the algorithms used have been shown through flowcharts in \nameref{chap:implementation}.
		\item Development of interfaces for full integration of {\GEM} in multiphysics simulations.
		\item Improved understanding of the modified quasichemical model (MQM), which is the most widely used thermodynamic model for treating short-range ordering such as in molten salts. See \nameref{chap:thermo}.
		\item Quantitative comparison of global optimisation algorithms for application in phase equilibria calculations and incorporation of these algorithms in \GEM. See \nameref{chap:implementation}.
		\item Software quality assurance  with rigorous source code control, continuous integration (CI) and verification to comply with NQA-I standard of MOOSE and MOOSE-based applications.
	\end{enumerate}

\section{Deliverables}
	The outcomes of this research were documented in several ways including publications in peer-reviewed journals, conference presentations and annual progress reports submitted to the Idaho National Laboratory. The main deliverables of this work were as follows:
	\begin{enumerate}
	\item Code base for {\GEM} delivered to Idaho National Laboratory's (INL) via export controlled repository on Nuclear Science User Facility (NSUF) High Performance Computing (HPC)'s GitLab instance.
	\item Annual progress reports submitted to INL in September 2019, September 2020, August 2021 and September 2022. 
	\item Four publications in peer-reviewed journals and one on global optimisation for phase equilibria, which is currently in preparation. See \nameref{publications}.
	\item Conference talks and posters presented at several conferences including NuFuel, Candu Fuels Conference and TMS Annual Meeting \& Exhibition. See \nameref{publications}.
	\item An internship at Idaho National Laboratory from June 2021 to September 2022, working on MOOSE integration of \GEM.
	\end{enumerate}
	