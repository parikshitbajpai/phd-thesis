\chapter{Goals of Research} \label{chapter_2}

	Facilitating the discovery and design of nuclear materials for innovative nuclear materials requires use of multiscale, multiphysics models and simulations and chemical thermodynamics forms an integral part of such simulations. The motivation behind this research is to directly integrate thermodynamic equilibrium computations in the Multiphysics Object Oriented Simulation Environment (\texttt{MOOSE}) which is a generic finite element method based simulation framework. By computing various thermodynamic values on a finite element mesh and using these values to inform macroscale and mesoscale models, an unprecedented level of sophistication can be achieved in multiscale, multiphysics models for materials.  

	Some of the challenges in achieving this objective include computational expense, predictive capabilities and reliability. First, relatively limited capabilities are currently available to perform such coupling. Within the \texttt{MOOSE} framework, the thermodynamic equilibrium code \texttt{Thermochimica} has only recently been integrated with the macroscale code \texttt{Bison} while the mesoscale code \texttt{Marmot} still relies on empirical correlations for thermodynamic inputs and coupling \texttt{Thermochimica} with it would require starting the process of coupling for scratch.  Second, the computational expense of thermodynamic equilibrium dramatically increases the computational burden of the multiphysics simulations. Finally, reliability is of utmost importance for simulation of nuclear materials and requires an enormous number of function evaluations. 
	
	While there are a number of existing thermodynamic equilibrium tools, all but a handful of them are standalone codes and since computational time for such computations is not of high importance, there has been no interest in optimising their performance. In order to maintain numerical stability, most of the codes restrict the number of system components but the intended application also requires that the new code be able to handle systems with a very large number of components. Another issue that needs to be addressed is the challenge associated with initialisation algorithms for the code. A far from accurate initial estimate can significantly reduce the convergence rate impeding the performance of the simulations and the current schemes used for initialisation often increase the computational cost when used at every iteration in the multiphysics simulations. Finally, the convergence of solution to the true solution and not a false positive must be ensured which requires the use of global optimisation methods. 
  
	The goal of this work is to develop a new state-of-the-art thermodynamic equilibrium code by leveraging the experience with the development and utilisation of \texttt{Thermochimica}. Though the thermodynamic equilibrium code is being developed within the corrosion modelling tool \texttt{Yellowjacket}, it would be easily couple-able with other applications such as \texttt{Bison} and \texttt{Rattlesnake}. The code will rely on the \texttt{MOOSE} framework and exploit the multitude of mathematical and development tools of the framework to ensure that the code meets the stringent requirements of the nuclear industry. It must be mentioned that while \texttt{MOOSE} and other \texttt{MOOSE} based applications solve systems of partial differential equations (PDE) using finite element method, thermodynamic equilibrium calculation is essentially a non-convex optimisation problem and requires much more developmental effort than many other \texttt{MOOSE} applications where essentially just new PDEs need to be implemented.
	
	The major contributions of the work would be as follows:
	\begin{enumerate}
		\item Development of a new advanced Gibbs energy minimiser written in C++ within the framework of \texttt{MOOSE} platform.
		\item Full integration within the multiphysics framework \texttt{MOOSE}, with the intent of coupling to the phase field code \texttt{Marmot}\footnote{The phase field module development and coupling is being investigated at University of Florida.}. 
		\item Enhanced initialisation algorithms to improve the computational performance.
		\item Investigation and implementation of robust global optimisation schemes to increase reliability and robustness.
		\item Software Quality Assurance  with rigorous verification and testing to comply with the NQA-1 guidelines required to be met for licensing.
	\end{enumerate}



