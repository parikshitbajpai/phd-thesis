\chapter*{Abstract}
\addcontentsline{toc}{chapter}{Abstract}
\setheader{Abstract}


Nuclear materials are highly complex multiscale, multiphysics systems and there has been an increasing interest in adopting a multiscale approach to better simulate the behaviour of nuclear materials. Material behaviour is influenced by many different physics, for example, mechanics such as dislocations, cracking, stress driven diffusion; chemistry such as corrosion, oxidation, reactive transport; heat conduction and species transport resulting in melting, precipitation; etc. Furthermore, the phenomena at the atomistic and micro scales drive the macro scale response. The multiscale modelling approach aims at using information from smaller scales to inform the models at increasing length scales and helps in effective prediction of performance of nuclear materials. The evolution of the composition of nuclear materials due to irradiation plays a significant role in driving the above mentioned phenomena and thermochemical equilibrium calculations help in providing the information such as material properties and boundary conditions for continuum and mesoscale codes. As a result, recent trends in modelling and simulation of nuclear materials have been to couple thermochemical equilibrium codes within multiphysics frameworks.

Many emerging nuclear technologies, such as the Molten Salt Reactor (MSR), use high temperature fluids such as molten fluoride/chloride salts, which lead to corrosion of the metal containment leading to problematic behaviours during reactor operations. Corrosion is an electrochemical process composed of oxidation and reduction reactions, which are driven by the thermodynamics and kinetics of the reactions. While thermodynamics determines whether or not a material can corrode, kinetics influences how quickly the material will corrode. This corrosion behaviour is also significantly affected by the material microstructure and predicting corrosion therefore requires a multiphysics approach that can couple quantitative electrochemistry models of corrosion and chemical reactions with thermochemical equilibrium computations. The Multiphysics Object Oriented Simulation Environment (MOOSE) developed by the Idaho National Laboratory provides a framework for multiphysics simulations but lacks the tools for predicting corrosion at the microstructure scale. Under Nuclear Energy Advanced Modelling and Simulation Program, a new MOOSE-based tool, YellowJacket, has been developed to perform such simulations and to predict quantities such as the rate of material loss, corrosion product production, and precipitate production in liquid.

As part of YellowJacket, a new thermochemical equilibrium solver, YellowJacket-GEM, was developed to quantitatively provide quantities of interest, for example chemical potential of stable species, by using the principles of thermodynamics. YellowJacket-GEM replaces many empirically based models for material properties and boundary conditions with a fundamentally sound approach that is founded on fundamental principles of thermodynamics while also reducing the computational cost associated with such coupling. The Gibbs energy minimiser uses state-of-the-art mathematical solvers and  global optimisation algorithms to reduce computational cost and improve accuracy of thermochemical equilibrium and is fully compatible with the MOOSE programming architecture in a cohesive manner using the same programming language and software libraries while satisfying the same software quality metrics and improving the thermodynamic capability. The coupling of YellowJacket-GEM with MOOSE phase field module to model corrosion in MSRs has also been performed and demonstrated.



\bigskip
\bigskip
\bigskip
\bigskip

\noindent
\textbf{Keywords:} Computational Thermodynamics; Gibbs Energy Minimisation; Mutiphysics; MOOSE; YellowJacket~-~GEM
