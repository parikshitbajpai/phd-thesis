\chapter*{Abstract}
\addcontentsline{toc}{chapter}{Abstract}
\setheader{Abstract}

Nuclear fuels and structural materials are highly complex systems which are remarkably challenging to understand and model. Material behaviour is influenced by many different phenomena such as mechanics (dislocations, cracking, etc.), chemistry (oxidation, reactive transport, corrosion, etc.), heat and mass transport, amongst  others. Moreover, lower scale phenomena inform and drive the phenomena at larger scales. The strong interactions between multiple physics at different length and time scales creates a need for multi-scale, multiphysics modelling tools. In nuclear fuels, and to a lesser extent in structural materials, the problem gets compounded by the fact that, in addition to an extreme environment, the composition of the system changes with time. For such complex systems, computational thermodynamics plays a key role in predicting many phenomena and is, at minimum, necessary for understanding and informing others. For this reason, there has been an increasing interest in incorporating equilibrium thermodynamics calculation in multiphysics frameworks such as Multiphysics Object Oriented Simulation Environment (MOOSE).

A phenomena of great interest in advanced reactors that employ high-temperature fluids, such as molten salts, is that of corrosion of fluid-facing structural materials. A new MOOSE-based tool, namely {\YJ}, has been developed to model such corrosion and the primary objective of this work is to develop a new equilibrium thermodynamic solver to provide thermodynamic material properties and boundary conditions for {\YJ} and other MOOSE-based simulation codes. While several thermodynamics codes already exist, the new software, called {\GEM}, aims to bring native equilibrium thermodynamic capability to the MOOSE framework and aims to address several concerns such as computational performance, limitations on system size and models, and software quality assurance.

{\GEM} exploits the fundamental laws of thermodynamics and uses state-of-the-art numerical solvers and global optimisation algorithms to accurately and efficiently model and predict phase evolution in nuclear materials. By using the advanced solver capabilities of Portable, Extensible Toolkit for Scientific Computation (PETSc) and coupling them with comprehensive global optimisation algorithms, the rate of convergence and robustness of the equilibrium calculations has been enhanced. The equilibrium thermodynamics solver has also been coupled with the MOOSE Phase Field module to improve the fidelity of molten-salt corrosion models.


%======================= Version 2 =======================
%Nuclear fuels and structural materials are highly complex systems that have traditionally been challenging to understand and model. Material behaviour is influenced by many different phenomena such as mechanics (dislocations, cracking, etc.), chemistry (oxidation, reactive transport, corrosion, etc.), heat and mass transport, amongst many others. Moreover, lower length scale phenomena inform and drive the phenomena at larger scales. The strong interactions between multiple physics at different length and time scales creates a need for multi-scale, multiphysics modelling tools which aim to use lower scale information to inform the models at larger scales with the goal of improving the predictive performance of models and simulations. In nuclear fuels, the problem gets compounded by the fact that, in addition to extreme environment, nuclear fission leads to formation of numerous fission products and the composition of the system changes with time. Transmutations and other phenomena lead to similar evolution of composition in structural materials as well, albeit to a lesser extent. For such complex systems, computational thermodynamics plays a key role in predicting many phenomena and is, at minimum, necessary for understanding and informing others. For this reason, there has been an increasing interest in incorporating equilibrium thermodynamics calculation in multiphysics frameworks such as Multiphysics Object Oriented Simulation Environment (MOOSE).
%
%A phenomena of particular interest in advanced reactors that employ high-temperature fluids such as molten salts is that of corrosion of fluid facing structural materials. Since corrosion is an electrochemical process occurring at the microstructural level and is driven by the thermodynamics and redox kinetics of the system, a new MOOSE-based tool, namely {\YJ}, has been developed to predict quantities such as the rate of material loss, corrosion product production, and precipitate production in liquid. The primary objective of this work is to develop a new equilibrium thermodynamic solver to provide thermodynamic material properties and boundary conditions for {\YJ} and other MOOSE-based simulation codes. While several thermodynamics codes already exist, the new software, called {\GEM}, aims to bring native equilibrium thermodynamic capability to the MOOSE framework and at the same time aims to address several concerns related to the current codes, specifically, computational performance, limitations on system size and models, and software quality assurance.


%======================= Version 1 =======================
%Many emerging nuclear technologies, such as the Molten Salt Reactor (MSR), use high temperature fluids such as molten fluoride/chloride salts, which lead to corrosion of the metal containment leading to problematic behaviours during reactor operations. Corrosion is an electrochemical process composed of oxidation and reduction reactions, which are driven by the thermodynamics and kinetics of the reactions. While thermodynamics determines whether or not a material can corrode, kinetics influences how quickly the material will corrode. This corrosion behaviour is also significantly affected by the material microstructure and predicting corrosion therefore requires a multiphysics approach that can couple quantitative electrochemistry models of corrosion and chemical reactions with thermochemical equilibrium computations. The Multiphysics Object Oriented Simulation Environment (MOOSE) developed by the Idaho National Laboratory provides a framework for multiphysics simulations but lacks the tools for predicting corrosion at the microstructure scale. Under Nuclear Energy Advanced Modelling and Simulation Program, a new MOOSE-based tool, YellowJacket, has been developed to perform such simulations and to predict quantities such as the rate of material loss, corrosion product production, and precipitate production in liquid.
%
%As part of YellowJacket, a new thermochemical equilibrium solver, YellowJacket-GEM, was developed to quantitatively provide quantities of interest, for example chemical potential of stable species, by using the principles of thermodynamics. YellowJacket-GEM replaces many empirically based models for material properties and boundary conditions with a fundamentally sound approach that is founded on fundamental principles of thermodynamics while also reducing the computational cost associated with such coupling. The Gibbs energy minimiser uses state-of-the-art mathematical solvers and  global optimisation algorithms to reduce computational cost and improve accuracy of thermochemical equilibrium and is fully compatible with the MOOSE programming architecture in a cohesive manner using the same programming language and software libraries while satisfying the same software quality metrics and improving the thermodynamic capability. The coupling of YellowJacket-GEM with MOOSE phase field module to model corrosion in MSRs has also been performed and demonstrated.

\bigskip
\bigskip
\bigskip
\bigskip

\noindent
\textbf{Keywords:} Computational Thermodynamics; Gibbs Energy Minimisation; Global Optimisation; {\YJ} Gibbs Energy Minimiser; Mutiphysics Object Oriented Simulation Environment (MOOSE)
