\chapter*{Acknowledgements}
\addcontentsline{toc}{chapter}{Acknowledgements}
\label{acknowledgements}

I am deeply grateful to my supervisor Prof. Markus Piro for not only giving me the opportunity to work this research but also for his constant guidance and encouragement. His contribution to this work and to my development as a researcher has been invaluable and his advice and enthusiasm has made this work extremely rewarding and delightful. I can't appreciate enough the support of Dr. Daniel Schwen who mentored me as an intern at the Idaho National Laboratory and from whom I have learnt innumerable things about MOOSE but more importantly on how to design and write good code.

This work has benefited from the numerous discussions with our collaborators, Chaitanya Bhave and Prof. Michael Tonks from the University of Florida. These discussions provided a great insight into the objectives and were always useful in deciding the direction we want to pursue with the development of the code and its coupling with the phase field code developed by Chaitanya. Along the same lines, I must also thank Dr. Ben Spencer, Prof. Ted Besmann, Dr. Jake McMurray and Dr. David Andersson, as the leaders of the NEAMS Chemistry and Corrosion technical area facilitated and guided this work. % I would like to thank Dr. Rich Martineau who initiated this work and Dr. David Andrs who supported the initial development of the code.

I wish to thank Prof. Lennaert van Veen and Prof. Hendrick de Haan who have provided constant help as members of the supervisory committee and also to the members of the examining committee Prof. Kirk Atkinson and Prof. Greg Lewis for taking time out of their very busy schedules to review this work. In particular, I would like to thank Prof. Maria Emelianenko (George Mason University) for very kindly agreeing to be the external examiner and for providing her valuable feedback on this work.

This work is an outcome of several years of research with numerous collaborations and inputs from many individuals and to acknowledge everyone would certainly be impossible. With that being said, I would like to start with Dr. Max Poschmann with whom I've worked throughout this research and who probably understands and shares a lot of my frustrations with figuring out models and finding bugs. Max was always available to discuss models, methods, code and a lot more. To all the members of the Computational Methods Development Team at INL who were always the most helpful and brilliant people to work with, I can't describe how much I've learnt from you all. To my colleagues and friends in both Nuclear Fuels and Material group and Collaborative Laboratory for Applied and Industrial Mathematics, I appreciate all our conversations about work and more, the lunches and coffee-breaks, staying well past midnights to work on assignments but most importantly for adding fun to work. Though I cannot name everyone here, I must mention Ernesto Geiger, Ksenia Lipkina, and Marcos Machado who started out as colleagues but are the greatest friends one can make.

I would like to thank my family without whom I would never be here. In particular, my mother, Vibha, my father, Vinod, and my sister, Sonakshi, who have all supported me, motivated me and inspired me. Lastly, I can't describe how much the encouragement and support of my partner, Dhayo, means to me. Thank you for always being there.


%Lastly, I would like to thank the many programs which made this work possible. This work was funded by the Department of Energy Nuclear Energy Advanced Modeling and Simulation program under Contract No. DE-AC07-05ID14517 with the US Department of Energy. This research made use of Idaho National Laboratory computing resources which are supported by the Office of Nuclear Energy of the U.S. Department of Energy and the Nuclear Science User Facilities under Contract No. DE-AC07-05ID14517. This research was undertaken, in part, thanks to funding from the Canada Research Chairs (950-231328) program of the Natural Sciences and Engineering Research Council of Canada.
